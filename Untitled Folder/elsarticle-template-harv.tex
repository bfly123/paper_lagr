%% 
%% Copyright 2007-2019 Elsevier Ltd
%% 
%% This file is part of the 'Elsarticle Bundle'.
%% ---------------------------------------------
%% 
%% It may be distributed under the conditions of the LaTeX Project Public
%% License, either version 1.2 of this license or (at your option) any
%% later version.  The latest version of this license is in
%%    http://www.latex-project.org/lppl.txt
%% and version 1.2 or later is part of all distributions of LaTeX
%% version 1999/12/01 or later.
%% 
%% The list of all files belonging to the 'Elsarticle Bundle' is
%% given in the file `manifest.txt'.
%% 
%% Template article for Elsevier's document class `elsarticle'
%% with harvard style bibliographic references

\documentclass[preprint,12pt,authoryear]{elsarticle}

%% Use the option review to obtain double line spacing
%% \documentclass[authoryear,preprint,review,12pt]{elsarticle}

%% Use the options 1p,twocolumn; 3p; 3p,twocolumn; 5p; or 5p,twocolumn
%% for a journal layout:
%% \documentclass[final,1p,times,authoryear]{elsarticle}
%% \documentclass[final,1p,times,twocolumn,authoryear]{elsarticle}
%% \documentclass[final,3p,times,authoryear]{elsarticle}
%% \documentclass[final,3p,times,twocolumn,authoryear]{elsarticle}
%% \documentclass[final,5p,times,authoryear]{elsarticle}
%% \documentclass[final,5p,times,twocolumn,authoryear]{elsarticle}

%% For including figures, graphicx.sty has been loaded in
%% elsarticle.cls. If you prefer to use the old commands
%% please give \usepackage{epsfig}

%% The amssymb package provides various useful mathematical symbols
\usepackage{amssymb}
%% The amsthm package provides extended theorem environments
%% \usepackage{amsthm}

%% The lineno packages adds line numbers. Start line numbering with
%% \begin{linenumbers}, end it with \end{linenumbers}. Or switch it on
%% for the whole article with \linenumbers.
%% \usepackage{lineno}

\journal{Nuclear Physics B}

\begin{document}

\begin{frontmatter}

\title{A multi-material HLLC Riemann solver with both elastic and plastic waves for 1D  elastic-plastic flows}

\author{Li Liu$^1$, Jun-bo Cheng $^{1,*}$, Jiequan Li $^{1}$}
%\cortext[mycorrespondingauthor]{
%Correspondence to: Junbo Cheng, Institute of Applied Physics and Computational Mathematics, Beijing 100094, China. E-mail: Cheng\_junbo@iapcm.ac.cn}

%\maketitle

\address{$^1$  Institute of Applied Physics and Computational Mathematics, Beijing 100094, China }

\begin{abstract}
  A multi-material HLLC-type  approximate Riemann solver with both elastic and plastic waves (MHLLCEP) is constructed for 1D elastic-plastic flows with the  hypo-elastic model and the von Mises yielding condition. Although Cheng in 2016 \cite{hllce} introduced a HLLC Riemann solver with elastic waves(HLLCE) for 1D elastic-plastic flows, Cheng assumed that pressure is continuous across the contact wave. This assumption maybe lead to big errors, especially for multi-material elastic-plastic flows. In our MHLLCEP, this assumption is not used again, and correspondingly the errors introduced by the assumption are deleted, describing and evaluating the plastic waves are more accurate than that in the HLLCE. Moreover, if the non-linear waves in the Riemann problem are only shock waves, even with the plastic waves, our MHLLCEP is theoretically accurate. For  the multi-material system, in this paper, a ghost cell method is used to achieve high-order spatial reconstruction across the interface without numerical oscillations. Based on the MHLLCEP, combining with the third-order WENO reconstruction method and the third-order Runge-Kutta method in time, a high-order cell-centered Lagrangian scheme for 1D multi-material elastic-plastic flows is built in this paper. A number of numerical experiments are carried out. Numerical results show  that the presented third-order scheme is convergent, robust, and essentially non-oscillatory. Moreover, for multi-material elastic-plastic flows, the scheme with  the MHLLCEP is more accurate and reasonable in resolving the multi-material interface than the scheme with the  HLLCE.
\end{abstract}

\begin{keyword}
  HLLC Riemann solver, high-order cell-centered Lagrangian scheme,  WENO scheme,  hypo-elastic model, elastic-plastic flows
\end{keyword}

\end{frontmatter}

%% Title, authors and addresses

%% use the tnoteref command within \title for footnotes;
%% use the tnotetext command for theassociated footnote;
%% use the fnref command within \author or \address for footnotes;
%% use the fntext command for theassociated footnote;
%% use the corref command within \author for corresponding author footnotes;
%% use the cortext command for theassociated footnote;
%% use the ead command for the email address,
%% and the form \ead[url] for the home page:
%% \title{Title\tnoteref{label1}}
%% \tnotetext[label1]{}
%% \author{Name\corref{cor1}\fnref{label2}}
%% \ead{email address}
%% \ead[url]{home page}
%% \fntext[label2]{}
%% \cortext[cor1]{}
%% \address{Address\fnref{label3}}
%% \fntext[label3]{}

%% \linenumbers

%% main text
\section{}
\label{}

%% The Appendices part is started with the command \appendix;
%% appendix sections are then done as normal sections
%% \appendix

%% \section{}
%% \label{}

%% If you have bibdatabase file and want bibtex to generate the
%% bibitems, please use
%%
%%  \bibliographystyle{elsarticle-harv} 
%%  \bibliography{<your bibdatabase>}

%% else use the following coding to input the bibitems directly in the
%% TeX file.

\begin{thebibliography}{00}

%% \bibitem[Author(year)]{label}
%% Text of bibliographic item

\bibitem[ ()]{}

\end{thebibliography}
\end{document}

\endinput
%%
%% End of file `elsarticle-template-harv.tex'.
